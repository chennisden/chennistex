%This is essentially the only way to reliably use loops, in the \makeatletter environment.
\makeatletter
\def\declarenumlist#1#2#3{%
\expandafter\edef\csname pgfmath@randomlist@#1\endcsname{#3}%
\count@\@ne
\loop
\expandafter\edef
\csname pgfmath@randomlist@#1@\the\count@\endcsname
  {\the\count@}
\ifnum\count@<#3\relax
\advance\count@\@ne
\repeat}

\declarenumlist{sollist}{1}{\value{solcounter}}

\def\prunelist#1{%
\expandafter\edef\csname pgfmath@randomlist@#1\endcsname
    {\the\numexpr\csname pgfmath@randomlist@#1\endcsname-1\relax}
\count@\pgfmath@randomtemp
\loop
\expandafter\let
\csname pgfmath@randomlist@#1@\the\count@\expandafter\endcsname
\csname pgfmath@randomlist@#1@\the\numexpr\count@+1\relax\endcsname
\ifnum\count@<\csname pgfmath@randomlist@#1\endcsname\relax
\advance\count@\@ne
\repeat}
\makeatother

% Print sols
\begin{enumerate}
\small
\itemsep2pt
\setcounter{sindex}{0}%
\whiledo{\value{sindex} < \value{solcounter}}{%
 \addtocounter{sindex}{1}%
 \pgfmathrandomitem\z{sollist}
 \begin{getsol}
 \z
 \end{getsol}
 \prunelist{sollist}
 }
\end{enumerate}